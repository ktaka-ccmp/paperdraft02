\section{Introduction}

(Background)
Recently, launching web services on cloud computing infrastructure is getting more popular.
Launching web services on the cloud is easier than launching them on on-premise data centers.
A web service on the cloud becomes scalable, which means it can accommodate a large amount of traffic from the Internet by increasing the number of web servers, on demand.
To distribute high volume traffic from the Internet to thousands of web servers load balancers are often used.
Major cloud providers have developed software load balancers[1,2] as part of their infrastructures, which they claim to have a high-performance level and scalability.
In the case of on-premise data centers, one can use proprietary hardware load balancers.
The actual implementation and the performance level of those existing load balancers are very different.

As for another issue to address regarding web services,
there are also needs to use multiple of cloud providers or on-premise data centers seamlessly, which spread across the world, to prepare for the disaster, to lower the cost or to comply with the legal requirement.
Linux container technology[3] facilitates these usages by providing container cluster management system as a middleware,
where one can deploy a web service that consists of a cluster of containers without modification even on different infrastructures.
However, in reality, it is difficult to do so, because existing load balancers are very different and are not included in the container cluster management system.
Therefore, users should always manually adjust their web services to the infrastructure.

(Problem)
In short, there exist several software load balancers that are specific to cloud vendors and on-premise data centers.
The differences among them are the major obstacles to provide uniform container cluster platform, which is necessary to realize web service migration across the different cloud providers and on-premise data centers.

(Proposal)
To address this problem, we propose a portable and scalable software load balancer that can be used in any environment including cloud providers and in on-premise data centers.
We can include this load balancer as a part of a container cluster management system, which acts as a common middleware on which web services run.
Users now do not need manual adjustment of their services to the infrastructures.
We will implement the proposed software load balancer using following technologies;
1) To make the load balancer usable in any environment, we containerize it using Linux container technology[3].
2) To make the load balancer scalable, we make it capable of being run in parallel using Equal Cost Multi-Path(ECMP) technique[4].
3) To make the load balancer’s performance level meet the need for 10Gbps network speed, we implement the load balancer using eXpress Data Plane(XDP) technology[5].
By providing container cluster management system including the proposed load balancer, we also aim to demonstrate migration capability, where one can migrate a web application across the different cloud providers and on-premise data centers, all over the world, with one push button.

(Summary of the results)
The outcome of our study will benefit users who want to deploy their web services on any cloud provider where no scalable load balancer is provided, to achieve high scalability.
Moreover, the result of our study will also benefit users who want to use a group of different cloud providers and on-premise data centers across the globe, as if it were a single computer on which their web services run.

(This paragraph needs rewrite)The rest of the paper is organized as follows.
Section \ref{Related Work} highlights work that deals specifically with container cluster migration, 
software load balancer containerization, and load balancer related tools within the context of the container technology. 
Section \ref{Architecture} will explain existing architecture problems and propose our solutions.
In Section \ref{IPVS}, experimental conditions and the parameters 
that we considered to be important in our experiment will be described in detail.
Then, we will show our experimental results and discuss the obtained performance characteristics in Section~\ref{XDP loadbalancer},  
which is followed by a summary of our work in Section~\ref{Conclusions}.

\section{Related Work}\label{Related Work}

Kubernetes

Docker swarm

IPVS

Google maglev

Microsoft Athena

Facebook shive

RFC Per flow ECMP  

XDP papers

\section{Load balancer and network architecture}\label{Architecture}
% \section{Traffice engineering for incomming traffic a cluster}\label{Traffice engineering}
The typical traditional on-premise data center uses a set of load balancers to distribute the incoming traffic to a cluster of servers thus providing scalability and redundancy of the service.
The traffic is routed from the outside of the infrastructure to one of the load balancers using active-backup redundancy protocol OSPF, VRRP or HSRP.

More recent architecture enable active-active redundancy using ECMP. The maglev[ref], athena[ref] and shive[ref] leverage this architecture by using software load balancers, thus providing scalability with less cost than the case where proprietary hardware load balancers are used.

However, since none of these load balancers are open sourced as of the writing of this paper, users can not use either of them, in the arbitrary cloud providers and in their on-premise data centers.
So providing a software load balancer that is usable in all of the cloud provider and in on-premise data centers is critical, which is discussed in Section~\ref{IPVS} and Section~\ref{XDP loadbalancer}.

Here the author discusses the possible network architectures, using such software load balancers.

(Here I explain using Figs....)

It should be noted that even if the load balancer becomes portable and scalable, the actual traffic a data center infrastructure can accommodate is ultimately limited by the incoming network bandwidth and by the core router.

\section{Per flow ECMP}\label{IPVS}
(Maybe cut this section )
Discuss routing traffic from upstream L3 switch. 
It is important connect to upstream switches using standard protocol.

ECMP  VRRP


\section{IPVS portable load balancer}\label{IPVS}

The author and his collegue proposed a portable load balancer for the Kubernetes cluster systems 
that is aimed at facilitating migration of container clusters for web services.
We implemented a containerized software load balancer that is run by Kubernetes as a part of container cluster, 
using Linux kernel's IPVS, as a proof of concept.
In order to discuss the feasibility of the proposed load balancer, we built 
a Kubernetes cluster system and conducted performance measurements.
Our experimental results indicate that the IPVS based load balancer in container improves the portability of 
the Kubernetes cluster system while it shows the similar performance levels as the existing iptables DNAT based load balancer.
We also clarified that choosing the right operating modes of overlay networks is important for the performance of load balancers. 
For example, in the case of flannel, only the vxlan and udp backend operation modes could be used 
in the cloud environment, and the udp backend significantly degraded their performance.
Furthermore, we also learned that the distribution of packet processing among multiple CPUs was very important
to obtain the maximum performance levels from load balancers.
%

The limitations of this work that authors aware of include the followings: 
1) We have not discussed the load balancer redundancy. 
Routing traffic to one of the load balancers while keeping redundancy in the container environment is a complex issue,
because standard Layer 2 rendandacy protocols, e.g. VRRP or OSPF\cite{moy1997ospf} that uses multicast, can not be used in many cases.
Further more, providing uniform methods independent of various cloud environments and on-premise datacenter is much more difficult.   
2) Experiments are conducted only in a 1Gbps network environment.
The experimental results indicate the performance of IPVS may be limited by the network bandwidth, 1Gbps, in our experiments. 
Thus, experiments with the faster network setting, e.g. 10Gigabit ethernet, are needed to investigate the feasibility of the proposed load balancer.
3) We have not yet compared the performance level of proposed load balance with those of cloud provider's load balancers.
It shoud be fair to compare the performance of proposed load balancer with those of the combination of the cloud load balancer and the iptables DNAT. 
The authors leave these issues for future work and they will be discussed elsewhere.


Even though we could successfully containerised the load balancers and could deploy it as a part of web service, the performance we obtained was not equivalent that of scalable load balancers provided by the GCP.

The next section will disscuss how to improve the performance levels of the open source software load balancer. 

\section{XDP loadbalancer}\label{XDP loadbalancer}

The ipvs is very dependent on the Netfilter and Linux kernel network stack.
The IP packet processing of the Linux kernel has been claimed to be inefficient, and thus unable to meet the speed requirement of 10Gbps and above.

Several alternative ways, DPDK, netmaps, PF\_RING and Maglev to increase the speed of the packet processing have been proposed.

Most bypass the Linux kernel network stack and process the packet in user spaces.
While they may improve the performance level in specific applications, they all have some issues regarding compatibility because of the bypass.
Often they require dedicated physical NIC, other than the ones used for standard Linux services, e.g., ssh.
This table summarizes, pros and cons of the proposed techniques to improve the speed of the packet processing.

(Table)

Recently, the Linux kernel introduced eXpress Data Plane (XDP)[ref] a novel way to improve the traffic manipulation speed, while keeping compatibility with the other functions of the Linux standard networking stack.

By using the XDP infrastructure, a user can write code that manipulates traffic in the very early stage of accepting the packets from outside the Linux box.
It will enable a user to inject a bytecode into the kernel, let the kernel compile it into native machine code and then the code will manipulate the traffic only if it matches the predefined conditions.
Traffic that did not match the conditions are pass to the Linux kernel's standard networking stack.

The advantage of XDP can be summarized as follows:
1)The traffic manipulation is very fast since it can tap the very early stage of packet processing flow.
2)Since it only affects the traffic that matches the predefined conditions, irrelevant traffic is processed by the Linux standard networks stack as usual.
3)The XDP let the users manipulate the traffic in the kernel space while keeping the safety.

The XDP code can be written in C code, compiled into bytecodes using clang compiler and loaded into kernel whenever a user has needs.
The XDP infrastructure provides built-in protection against dangerous codes.
In contrast, while one can always implement the Linux kernel modules that manipulate the traffic, it often crashes the kernel when something wrong happens.
Thus, the XDP provide in-kernel traffic manipulation capability while keeping the safety, without affecting the standards network services.

In the course of the study, the author comes to believe that it is critical to provide a software load balancer that is faster than the existing ipvs.
Here he discusses the design and prototype implementation of such a load balancer.





(The author started to write a prototype code of XDP load balancer.
Here the design and the prototype implementation are presented.)

(Comparison XDP, DPDK, ipvs, iptables.

DPDK: pros: bypass kernel network stack cons: only for Intel NIC, dedicated NIC required
XDP: pros: hooks to the NIC driver redirect
)

\section{Conclusion}\label{Conclusion}

In order to realize the smooth migration of the container clusters, providing a uniform environment, i.e., uniform container cluster infrastructure on top of the base infrastructure is very important.
Providing a standard load balancer architecture for incoming traffic is critical to achieving that purpose.

The author investigated the general architecture of the load balancer and network configurations.
The lateral scalability using the software load balancer and ECMP is critical in order to meet the future demand of the large scale web services.

The author and colleague also investigated containerization of ipvs load balancer to improve the portability of the web services, using Kubernetes.
They revealed that this would improve the portability of the service while keeping the performance level of conventional architecture.

The author also started to implement a novel software load balancer using recently introduced Linux kernel's XDP infrastructure. While it is in the preliminary stage of the development, essential functions and design issues have been already clarified.

By realizing smooth migration capability across the different cloud providers and on-premise data centers, users are freed from vendor lock-ins and ultimately obtain the opportunity to deploy global scale web services.


\section{Future work}\label{Future work}

1. Performance measurement of XDP loadbalancer

2. Balancing algorithm especially consistent hashing
(Assessment: packet reordering, persistent ssl conncetion)
 
3. Containerization of XDP loadbalancer


(Future work, out of scope of the dissertation)
Federation of cluster infra. in different Cloud/DC.
Data Sync.
Global traffic routing

